\documentclass[a4paper]{etutez}
\usepackage{xkeyval}
\usepackage[turkish]{babel}
\usepackage[latin5]{inputenc}
\usepackage{graphicx}
\usepackage[toc,page,titletoc]{appendix}
\usepackage[T1]{fontenc}
\usepackage{mathtools}
\usepackage{epstopdf}
\usepackage{subfigure}
\usepackage{tabulary}
\usepackage{listings}
\usepackage{cite}
\usepackage[top=4cm, bottom=2.5cm, left=4cm, right=2.5cm]{geometry}
\usepackage{setspace}
\usepackage{enumerate}
\usepackage{amsmath}
\usepackage{bm}

\DeclareGraphicsExtensions{.pdf,.png,.jpg, .eps}
\DeclareMathOperator*{\argmin}{arg\,min}

\pagenumbering{roman}

\thesistype{Ph.D.} %Yuksek lisans icin M.Sc. doktora icin Ph.D.
\teztipi{Doktora}  %Tez tipini yazin

\keywords{Wireless sensor networks, bandwidth, interference, transmission power control, discrete power levels, power assignment strategies, optimization, mathematical programming, energy efficiency, network lifetime}
\anahtarsoz{Kablosuz alg�lay�c� a�lar, bant geni�li�i, giri�im, iletim g�c� kontrol�, ayr�k g�� seviyeleri, g�� atama stratejileri, optimizasyon, matematiksel programlama, enerji verimlili�i, a� ya�am s�resi}

\title{ANALYZING NETWORK LIFETIME OF WIRELESS SENSOR NETWORKS WITH MATHEMATICAL PROGRAMMING}
\baslik{KABLOSUZ ALGILAYICI A�LARDA A� YA�AM S�RES�N�N MATEMAT�KSEL PROGRAMLAMA �LE �NCELENMES�}  % buyuk harflerle yazilmali

\yazar{H�SEY�N �OTUK}    % buyuk harflerle yazilmali
\yazarkucuk{H�seyin �OTUK}    % Soyadi Buyuk harflerle yazilmali
\enstitu{FEN B�L�MLER�} % buyuk harflerle yazilmali
\enstitukucuk{Fen Bilimleri } % kucuk harflerle yazilmali
\institute{Institute of Natural and Applied Sciences}
\bolum{B�LG�SAYAR M�HEND�SL��� } % buyuk harflerle yazilmali
\bolumkucuk{Bilgisayar M�hendisli�i } % kucuk harflerle yazilmali
\dept{Computer Engineering}
\supervisor{Assoc. Prof. Kemal BI�AKCI}
\supervisorr{Assoc. Prof. B�lent TAVLI}
\tezyoneticisi{Do�. Dr. Kemal BI�AKCI}
\tezyoneticisii{Do�. Dr. B�lent TAVLI}
\juribaskani{Do�. Dr. Hakan G�LTEK�N}
\juriuyesi{Do�. Dr. Kemal BI�AKCI}
\juriuyesii{Do�. Dr. B�lent TAVLI}
\juriuyesiii{Do�. Dr. Ali KARA}
\juriuyesiiii{Yrd. Do�. Dr. Tansel �ZYER}
\anablmdalibsk{Do�. Dr. Erdo�an DO�DU}
\enstitumuduru{Prof. Dr. Necip CAMU��U}
\director{Prof. Dr. Necip CAMU��U}
\firstreader{Assoc. Prof. Kemal BI�AKCI}
\secondreader{Asst. Prof. Hakan G�LTEK�N}
\thirdreader{Asst. Prof. Tansel �ZYER}
\fourthreader{Assoc. Prof. B�lent TAVLI}
%\copyrightyear{2001}
\submitdate{December 2013}  % buyuk harflerle yazilmali
\tarih{ARALIK 2013}   % buyuk harflerle yazilmali
\tarihkucuk{Aral�k 2013}   % kucuk harflerle yazilmali


\newtheorem{thm}{Theorem}
\newtheorem{preexam}{Example}
\newenvironment{exam} {\begin{preexam}\rm}{\end{preexam}}
\newtheorem{lem}{Lemma}
\newtheorem{proprty}{Property}
\newtheorem{cor}{Corollary}
\newtheorem{defn}{Definition}
\newcommand{\pe}{\preceq}
\newcommand{\po}{\prec}


\hyphenpenalty=5000
\tolerance=1000

\begin{document}

%\titlepageMS   % Yuksek lisans tezi kapak sayfasi
\titlepagePhD    % Doktora tezi kapak sayfasi
%\setcounter{page}{2}
%\signaturepageMS  % Yuksek lisans tezi imza sayfasi
\signaturepagePhD     % Doktora tezi imza sayfasi
\tezbildirimsayfasi    % tez bildirim sayfasi


\begin{ozet}
�stlendikleri misyon ve tipik uygulamalarda g�r�len genel karakteristikleri itibariyle kablosuz alg�lay�c� d���mler, genelde k�s�tl� kaynaklara (kullan�lacak enerji ve bant geni�li�i miktar�, haberle�me mesafesi, hesaplama g�c� ve bellek miktar� gibi) sahiptir. Yar� iletken, a� ve malzeme bilimi teknolojilerindeki son geli�meler sayesinde donan�m ve maliyet ile ilgili k�s�tlar�n a��lmas�, ba�lang��taki enerji miktar�n� kablosuz alg�lay�c� a�larda en kritik kaynak haline getirmi�tir. Dolay�s�yla kablosuz alg�lay�c� a�lar konusunda g�n�m�ze kadar yap�lan �al��malar, enerji sarfiyat�n� d���rerek, ba�ka bir deyi�le enerji verimli ��z�mler �reterek a� ya�am s�resini artt�rmak konusunda yo�unla�m��t�r. Bu �al��mada kablosuz alg�lay�c� a�larda giri�im ve bant geni�li�inin a� ya�am s�resine etkileri incelenmi�, bant geni�li�i ihtiyac�n� belirleyen parametreler tespit edilerek belli parametreler alt�nda optimum ��z�m i�in gerekli minimum bant geni�li�i miktar� belirlenmi�tir. Ayr�ca, kablosuz alg�lay�c� a�larda ayr�k iletim g�c� kontrol� yap�ld��� durumda a� ya�am s�resi karakteristi�i incelenmi�, farkl� g�� atama stratejileri kullan�ld���nda a� ya�am s�resinde meydana gelen etkiler mercek alt�na al�nm��t�r. Kullan�lan g�� atama stratejileri a� ya�am s�resi ve bant geni�li�i ihtiya�lar� a��s�ndan birbiriyle kar��la�t�r�lm��t�r. Bu a�amada pozisyonlama hatalar�n�n ve olas�l�ksal radyo yay�l�m�n�n sonu�lara etkileri ayr�ca ele al�nm��t�r. �nceki �al��malarda yer alan algoritma tabanl� analiz veya deneysel �l��mlerden farkl� olarak bu �al��mada �e�itli matematiksel programlama y�ntemleri kullan�larak tasarlanan modeller sayesinde, geni� bir yelpazede b�y�k bir parametre k�mesi ile detayl� incelemeler yap�lm��t�r. Al�nan sonu�lar, bant geni�li�inin a� ya�am s�resi �zerinde belli bir aral�kta etkili oldu�unu, giri�imin ihtiya� duyulan bant geni�li�i miktar�n� artt�rd���n�, bant geni�li�i gereksiniminin bir �ok a� ve sistem parametresine ba�l� oldu�unu g�stermi�tir. Di�er taraftan ayr�k iletim g�c� kontrol�n�n s�rekli duruma k�yasla a� ya�am s�resinde azalmaya neden oldu�u, iletim g�c�n�n daha hassas ayarlanabildi�i modellerde daha iyi a� ya�am s�resi elde edilirken, ayn� zamanda daha fazla bant geni�li�ine ihtiya� duyuldu�u g�r�lm��t�r. Pozisyonlama hatalar�n� tolere etmek i�in iletim g�c�n�n artt�r�lmas� ile, g�� kontrol�ndeki ayr�kla�t�rma seviyesinden kaynaklanan enerjideki kay�plar�n azald��� g�zlenmi�tir. Buna ra�men y�ksek pozisyonlama hatalar� g�r�ld���nde bile, iletim g�c�n�n daha hassas ayarlanabilmesi halinde daha iyi a� ya�am s�resi elde edilebildi�i anla��lm��t�r. Ayr�k iletim g�c� kontrol�nde yol kayb�ndan kaynaklanan fazla enerji kullan�m�n�n paket alma oran�n� artt�rmas� nedeniyle baz� durumlarda uygun paket alma oran� hedeflenerek s�rekli durumdan daha iyi sonu�lar al�nabilece�i g�r�lm��t�r. 
\end{ozet}


\begin{abstract}
According to the needs of typical applications, wireless sensor nodes are designed to be low-cost, small-sized, and energy-efficient devices. In order to satisfy these production requirements, they generally have scarce resources like energy, bandwidth, communication range, processing power, and memory. After the limitations related to cost and hardware are met by the progress on semiconductor, network, and materials technologies; energy becomes the most critical resource for Wireless Sensor Networks (WSNs). So, recent studies about WSNs concentrated on developing energy-efficient solutions to optimize network lifetime. However, most of these studies ignored the effect of finite bandwidth and discretization of transmission power on WSNs. On the other hand, there are different types of transmission power assignment strategies studied so far, and while it is evident that more fine-tuned power assignment improves network lifetime, the net impact of these strategies remains unclear. In this study, we develop novel mathematical programming frameworks which enable us not only to examine the effects of limited bandwidth and discrete transmission power control, but also to quantify the impact and make a systematic comparison of various power assignment strategies. We analyze the network bandwidth from several aspects with various system parameters. In order to obtain optimal network lifetime with specific parameters, we expose the methodology to determine the minimum amount of required bandwidth. We also investigate the effects of the granularity of power levels on energy dissipation characteristics. Different types of power assignment strategies are analyzed by using two sets of experimental data to compare the performance of these strategies in terms of network lifetime and link bandwidth. In order to see the effects of probabilistic radio propagation, widely used log-normal shadowing path loss model is adopted into existing models. Our results show that, link bandwidth affects network lifetime within a specific range. As interference rises, the amount of required bandwidth to obtain optimum lifetime increases as well. The granularity of discrete energy consumption has a profound impact on WSN lifetime and when discretization level ascends, network lifetime generally lessens. Results expose that while more fine-grained control of transmission power improves network lifetime, it also requires much more bandwidth. 
\end{abstract}

\begin{tesekkur}
�al��malar�m  boyunca  de�erli yard�m ve katk�lar�yla beni y�nlendiren k�ymetli hocalar�m Do�. Dr. Kemal BI�AKCI ve Do�. Dr. B�lent TAVLI'ya, yine �nemli tecr�belerinden faydaland���m TOBB Ekonomi ve Teknoloji �niversitesi End�stri M�hendisli�i B�l�m� ��retim �yelerinden Do�. Dr. Hakan G�LTEK�N'e, k�smi numerik analizler i�in hesaplama kaynaklar�ndan faydaland���m T�B�TAK ULAKB�M Y�ksek Ba�ar�ml� ve Grid Hesaplama Merkezi'ne, verdikleri k�smi burs ile maddi destek sa�layan T�B�TAK B�DEB'e, ara�t�rma g�revlisi arkada��m Davut �NCEBACAK'a, son olarak benden desteklerini hi�bir zaman esirgemeyen e�im Seda'ya, anneme ve babama te�ekk�r� bir bor� bilirim.
\end{tesekkur}


\pagestyle{plain}

\makeatother


%\thispagestyle{plain}
%\pagestyle{plain}

\tableofcontents
\newpage
\newpage

\addcontentsline{toc}{chapter}{\numberline{�EK�L L�STES�}}  
\listoffigures  % Eger tezde herhangi bir sekil yoksa silinmelidir 
\newpage
\newpage
\addcontentsline{toc}{chapter}{\numberline{TABLO L�STES�}}  		  
\listoftables % Tezde herhangi bir tablo yoksa silinmelidir

\newpage
\newpage
\newpage
\pagestyle{plain}
\addcontentsline{toc}{chapter}{\numberline{KISALTMALAR}}
\begin{center}
{\LARGE \bf KISALTMALAR}
\end{center}

\begin{tabular}{ll} 
\textbf{K�saltma} &			\textbf{A��klama} \\
KAA &  Kablosuz alg�lay�c� a� \\
DSN &  Ayr�k alg�lay�c� a�lar (distributed sensor networks) \\
M�B &  Merkezi i�lem birimi (central processing unit) \\
ADC &  Analog say�sal �evirici (analog digital converter) \\
PRR &  Paket alma oran� (packet reception rate) \\
RSSI &  Radyo sinyal seviyesi g�stergesi (radio signal strength indicator) \\
LQI &  Ba�lant� kalite g�stergesi (link quality indicator) \\
GPS &  Yer bulma sistemi (global positioning system) \\
MAC &  Ortam eri�im kontrol� (media access control) \\
MP &  Matematiksel programlama (mathematical programming) \\
DP &  Do�rusal programlama (linear programming) \\
TP &  Tamsay� programlama (integer programming) \\
�TP &  �kili tamsay� programlama (binary integer programming) \\
KTP &  Karma tamsay� programlama (mixed integer programming) \\
KTDP &  Karma tamsay� do�rusal programlama (mixed integer linear \\ & programming) \\
HCB &  �smini kendisini tasarlayan yazarlar�n (Heinzelman, Chandrakasan,\\ & Balakrishnan) isimlerinin ba�harflerinden alan enerji modeli\\
LNS &  Log-normal g�lgeleme (log-normal shadowing) \\
PNM-SL &  A� genelinde g�� atamas� yap�labilen tek seviyeli Mica modeli \\ & (Per Network Mica Model Single Level) \\
PLM-ML &  Ba�lant� baz�nda g�� atamas� yap�labilen �ok seviyeli Mica modeli\\ &  (Per Link Mica Model Multiple Level) \\
PNM-ML &  A� genelinde g�� atamas� yap�labilen �ok seviyeli Mica modeli\\ &  (Per Network Mica Model Single Level) \\
PSM-ML &  D���m baz�nda g�� atamas� yap�labilen �ok seviyeli Mica modeli\\ &  (Per Network Mica Model Single Level) \\
PNM-SL-PL &  A� genelinde g�� atamas� yap�labilen tek seviyeli yol kay�pl� Mica \\ & modeli (Per Network Mica Model Single Level) \\
PLM-ML-PL &  Ba�lant� baz�nda g�� atamas� yap�labilen �ok seviyeli yol kay�pl� \\ & Mica modeli (Per Link Mica Model Multiple Level) \\
PNM-ML-PL &  A� genelinde g�� atamas� yap�labilen �ok seviyeli yol kay�pl� Mica \\ & modeli (Per Network Mica Model Single Level) \\
PSM-ML-PL &  D���m baz�nda g�� atamas� yap�labilen �ok seviyeli yol kay�pl� \\ & Mica modeli (Per Network Mica Model Single Level) \\
\end{tabular}
\newpage
\pagestyle{plain}
\addcontentsline{toc}{chapter}{\numberline{SEMBOL L�STES�}}
\begin{center}
{\LARGE \bf SEMBOL L�STES�}
\end{center}
\begin{tabular}{ll} 
\textbf{Sembol} &			\textbf{A��klama} \\
$G$ &  D���mlerin ve veri ak��lar�n�n olu�turdu�u a�� ifade eden �izge \\ 
$V$ &  Baz istasyonu dahil a�daki d���mlerin k�mesi \\ 
$W$ &  Baz istasyonu hari� a�daki d���mlerin k�mesi \\ 
$A$ &  D���mler aras�ndaki veri ak��lar�n�n k�mesi \\ 
$t$ &  A� ya�am s�resi \\ 
$N$ &  D���m say�s� \\ 
$f_{ij}$ &  \emph{i}  d���m�nden \emph{j} d���m�ne veri ak��� \\ 
$g^{l}_{ij}$ &  \emph{i}  d���m�nden \emph{j} d���m�ne \emph{l}. g�� seviyesinde veri ak���\\ 
$s_i$ &  Bit cinsinden \emph{i} d���m�nde �retilen veri miktar� \\ 
$d_{int}$ & 1-boyutlu topolojide d���mler aras�ndaki mesafe \\ 
$d_{ij}$ & \emph{i} ve \emph{j} d���mleri aras�ndaki mesafe \\ 
$d^e_{ij}$ & \emph{i} ve \emph{j} d���mleri aras�nda hesaplanan mesafe \\ 
$d^c_{ij}$ & \emph{i} ve \emph{j} d���mleri aras�nda hesaplanm�� d�zeltilmi� mesafe \\ 
$R_{net}$ & 2-boyutlu topolojide a��n yay�l�m y�zeyinin yar��ap� \\ 
$t_r$ & �letim menzili \\
$\rho$ & Elektronik devrede harcanan enerji \\ 
$\varepsilon$ & Vericinin verimlili�i \\ 
$e_i$ & Her d���m�n bataryas�nda ba�lang��ta bulunan enerji miktar� \\ 
$\alpha$ & Yol kayb� katsay�s� \\ 
$\eta$ & En k���k ayarlanabilir enerji miktar� \\ 
$P_{rx}$ & Bir bit almak i�in harcanan g�� miktar� \\ 
$\linebreak P_{tx}(d_{ij}),dB$ & Desibel cinsinden \emph{i} d���m�nden \emph{j} d���m�ne bir bit g�ndermek \\ & i�in harcanan g�� miktar�\\ 
$\linebreak P_{tx}(d_{ij}),W$ & Watt cinsinden \emph{i} d���m�nden \emph{j} d���m�ne bir bit g�ndermek \\ & i�in harcanan g�� miktar�\\ 
$E_{rx}$ & Bir bit almak i�in harcanan enerji miktar� \\ 
$\linebreak E_{tx}(d_{ij})$ & \emph{i} d���m�nden \emph{j} d���m�ne bir bit g�ndermek i�in harcanan \\ & enerji miktar�\\ 
$\linebreak E^{C-HCB}_{rx}$ & S�rekli \emph{HCB} modelinde bir bit almak i�in harcanan enerji \\ 
$\linebreak E^{C-HCB}_{tx}(d_{ij})$ & S�rekli \emph{HCB} modelinde \emph{i} d���m�nden \emph{j} d���m�ne bir bit \\
 & g�ndermek i�in harcanan enerji \\ 
$\linebreak E^{D-HCB}_{rx}$ & Ayr�k \emph{HCB} modelinde bir bit almak i�in harcanan enerji \\ 
$\linebreak E^{D-HCB}_{tx}(d_{ij})$ & Ayr�k \emph{HCB} modelinde \emph{i} d���m�nden \emph{j} d���m�ne bir bit \\
 & g�ndermek i�in harcanan enerji \\ 
$\linebreak E^{C-HCB-LNS}_{rx}$ & Yol kay�pl� s�rekli \emph{HCB} modelinde bir bit almak i�in harcanan \\ & enerji \\ 
$\linebreak E^{C-HCB-LNS}_{tx}(d_{ij})$ & Yol kay�pl� s�rekli \emph{HCB} modelinde \emph{i} d���m�nden \emph{j} d���m�ne \\ &bir bit g�ndermek i�in harcanan enerji \\ 
\end{tabular}

\begin{tabular}{ll} 
$\linebreak E^{D-HCB-LNS}_{rx}$ & Yol kay�pl� ayr�k \emph{HCB} modelinde bir bit almak i�in harcanan \\ & enerji \\ 
$\linebreak E^{D-HCB-LNS}_{tx}(d_{ij})$ & Yol kay�pl� ayr�k \emph{HCB} modelinde \emph{i} d���m�nden \emph{j} d���m�ne \\
 & bir bit g�ndermek i�in harcanan enerji \\ 
$S_L$ & G�� seviyeleri k�mesi \\ 
$N_L$ & $S_L$ k�mesinde ayr�k g�� seviyesi say�s�\\ 
$\linebreak t^M_{r}(l)$ & Mica modelinde \emph{l}. g�� seviyesindeki iletim menzili\\ 
$\linebreak E^{M}_{tx}(l)$ & Mica modelinde \emph{l}. g�� seviyesindeki bit ba��na harcanan iletim \\ & enerjisi\\ 
$E^{M}_{rx}$ & Mica modelinde bir bit almak i�in harcanan enerji \\ 
$\linebreak E^{M-opt}_{tx}(d_{ij})$ & Mica modelinde optimum g�� seviyesinde \emph{i} d���m�nden \emph{j} \\ & d���m�ne bir bit g�ndermek i�in kullan�lan harcanan enerji \\
$l_{min}$ & En k���k g�� seviyesi \\ 
$l_{max}$ & En b�y�k g�� seviyesi \\ 
$l_{PNM-SL}$ & Verilen \emph{PNM-SL} modelini ��zmek i�in kullan�lan g�� seviyesi \\ 
$L_{PNM-ML}$ & \emph{PNM-ML} modelinde mevcut g�� seviyesi say�s�n�n �st limiti\\ 
$L_{PSM-ML}$ & \emph{PSM-ML} modelinde mevcut g�� seviyesi say�s�n�n �st limiti\\ 
$h^l$ & \emph{l} g�� seviyesinde t�m d���mler taraf�ndan iletilen toplam veri \\ & miktar�\\ 
$a^l$ & Karar de�i�keni ($h^l$ "0"dan farkl� ise 1, de�ilse 0) \\ 
$h^l_i$ & \emph{l} g�� seviyesinde \emph{i} d���m�nden ��kan toplam veri miktar�\\ 
$a^l_i$ & Karar de�i�keni ($h^l_i$ "0"dan farkl� ise 1, de�ilse 0)\\ 
$M$ & Yeterince b�y�k bir sabit \\ 
$B$ & Ba�lant� bant geni�li�i \\ 
$\linebreak I^i_{jk}$ & \emph{j} d���m� \emph{k} d���m� ile haberle�irken \emph{i} d���m� �zerinde \\ & meydana gelen ve giri�imden kaynakl� veri ak��� \\ 
$\gamma$ & Giri�im fakt�r� \\ 
$d_0$ & Yol kayb� modelinde �nceden belirlenmi� referans uzakl�k \\
$PL(d_{ij}),dB$ & Desibel cinsinden \emph{i} d���m� ile \emph{j} d���m� aras�ndaki yol kayb� \\ 
$PL(d_{0}),dB$ & Desibel cinsinden referans mesafenin yol kayb� \\ 
$P_n,dB$ & Desibel cinsinden g�r�lt�n�n taban de�eri \\ 
$\sigma$ & Standart sapma \\
$\varphi$ & Paket b�y�kl��� \\ 
$X_{\sigma},dB$ & Desibel cinsinden $\sigma$ standart sapmaya sahip s�f�r-ortalamal� \\ & Gauss rastgele de�i�keni \\
$\chi_{ij}$ & \emph{i} d���m� ile \emph{j} d���m� aras�nda ger�ekle�en paket alma oran� \\ 
$\chi_{trg}$ & Hedeflenen paket alma oran� \\ 
$\psi(d_{ij})$ & $d_{ij}$ mesafesindeki sinyalin g�r�lt�ye oran� \\ 
$\lambda_{ij}$ & \emph{i} d���m� ile \emph{j} d���m� aras�nda paketin yeniden g�nderilme \\ & say�s� \\ 
\end{tabular}

\begin{tabular}{ll} 
$e_{max}$ & Maksimum pozisyonlama hatas� \\
$t_{rnd}$ & Zaman�n a� ya�am s�resi boyunca b�l�nd��� e�it zaman dilimi \\
$T_{b}$ & Bir bitin iletilmesi i�in gereki zaman \\
$x$ & A� ya�am s�resinin b�l�nd��� zaman dilimlerinden her biri \\
$g(i,j,x)$ & $x$ zaman diliminde $i$ d���m�nden $j$ d���m�ne veri ak���n� g�steren \\ & ikili de�i�ken \\
$s(i,x)$ & $x$ zaman diliminde $i$ d���m�nde veri �retimini ifadeeden ikili de�i�ken \\
$c(i,j,k,l)$ & $(i,j)$ ile $(k,l)$ ba�lant�lar�n�n �ak��ma durumunu g�steren matris \\
$PS$ & Paket b�y�kl��� \\
$NP$ & G�nderilecek paket say�s� \\
$NS$ & Zaman dilimi say�s� \\
\end{tabular}  % Tezin basina kisaltmalar ya da semboller sayfasi konacaks basindaki '%' kaldirilmeli ve semboller.tex dosyasi olusturulmalidir 
\newpage

\setcounter{secnumdepth}{5}
\setcounter{tocdepth}{4}


\thispagestyle{plain}
\pagestyle{plain}

\pagenumbering{arabic}
\setcounter{page}{1}
\setcounter{section}{1}

\onehalfspacing

\chapter{G�R��} \label{chap:giris}

�rnek metin 

\section{�al��man�n Amac�} \label{sec:amac}

�rnek metin 

\section{�al��man�n �nemi} \label{sec:onem}

�rnek metin 
\section{Problem Tan�m�} \label{sec:problem}

�rnek metin 

\section{S�n�rl�l�klar} \label{sec:sinir}

�rnek metin 

\section{Varsay�mlar} \label{sec:varsayim}

�rnek metin 

\section{Katk�lar} \label{sec:katkilar}

�rnek metin 
\chapter{�LG�L� L�TERAT�R} \label{chap:lit}

�rnek metin 
\chapter{KAVRAMSAL �ER�EVE} \label{bolum3}

�rnek metin 
\chapter{S�STEM MODEL�} \label{bolum4}

�rnek metin 
\chapter{BULGULAR VE DE�ERLEND�RMELER}

�rnek metin 
\chapter{SONU�LAR VE �NER�LER} \label{sec:results}

�rnek metin 

%\newpage
\pagestyle{plain}
\addcontentsline{toc}{chapter}{\numberline{KAYNAKLAR}}


\begin{thebibliography}{99}

\bibitem{akturk} Akturk, M.S., An exact tool allocation approach for CNC machines. {\em International Journal of Computer Integrated Manufacturing}, {\bf 12} (2): 129--140, 1999.

\bibitem{hakan_2005} Gultekin, H., Akturk, M.S., Karasan,O.E., Robotic cell scheduling with operational flexibility. {\em Discrete Applied Mathematics}, {\bf 145} (3): 334--348, 2005.

\end{thebibliography} % Kaynakca .bib seklinde degil de kaynakca.tex dosyasi olarak hazirlanirsa bu satir kullanilmali
\bibliographystyle{ieeetr} % kaynakca.bib ile beraber kullanim icin.
\bibliography{kaynakca} % kaynakca.bib dosyasi hazirlanirsa bu satir kullanilmali
\renewcommand{\appendixname}{}
\renewcommand{\appendixtocname}{EKLER}
\renewcommand{\appendixpagename}{EKLER}

\begin{appendices}
\appendix
\chapter{�rnek GAMS Kodu}

�rnek metin 

\chapter{�rnek MATLAB Grafik �izdirme Kodu}

�rnek metin 

\end{appendices}
\newpage
\pagestyle{plain}
\addcontentsline{toc}{chapter}{\numberline{�ZGE�M��}}
\begin{center}
{\LARGE \bf �ZGE�M��}
\end{center}
\vspace{0.05cm}
{\bf K���SEL B�LG�LER}


\noindent
\begin{tabular}{@{}lll@{}}
Soyad�, Ad� & : �OTUK, H�seyin &\\
Uyru�u & : T.C.&\\
Do�um tarihi ve yeri & : 23.06.1980 �stanbul&\\
Medeni hali & : Evli&\\
Telefon & : 312 298 9326&\\
Faks & : 312 298 9393&\\
e-mail & : hcotuk@etu.edu.tr &\\
\end{tabular}

\vspace{0.05cm}
\noindent
{\bf E��T�M}


\noindent
\begin{tabular}{@{}llc@{}}
{\bf Derece} & {\bf E\u{g}itim Birimi} & {\bf Mezuniyet Tarihi}\\
Doktora & TOBB Ekonomi ve Teknoloji \"Universitesi & 2013\\
Y. Lisans & TOBB Ekonomi ve Teknoloji \"Universitesi & 2008\\
Lisans & S�leyman Demirel �niversitesi & 2002\\
\end{tabular}

\vspace{0.05cm}
\noindent
{\bf �LG� ALANLARI}

\noindent
\begin{tabular}{@{}lllll@{}}
Bulut Bili�im &		& A� Teknolojileri &	& Bilgi G�venli�i \\
Paralel Programlama &	& G�m�l� Sistemler &	& Optimizasyon \\
Sanalla�t�rma &	& HPC & 	& Ye�il BT Altyap�lar� \\
\end{tabular}

\vspace{0.05cm}
\noindent
{\bf �� DENEY�Mi}


\noindent
\begin{tabular}{@{}lll@{}}
{\bf Y{\i}l} & {\bf Yer} & {\bf G\"orev}\\
2012- & T�B�TAK ULAKB�M & Ba�uzman Ara�t�rmac�\\
2006-2012 & TOBB Ekonomi ve Teknoloji \"Universitesi & Bili�im Teknolojileri M�d�r�\\
2004-2006 & TOBB Ekonomi ve Teknoloji \"Universitesi & A� ve Sistem Y�neticisi\\
2003-2004 & Terra Elektronik Ltd. �ti. & Yaz�l�m M�hendisi\\
\end{tabular}

\vspace{0.05cm}
\noindent
{\bf YABANCI D�L}

\noindent
\begin{tabular}{@{}l@{}}
�ngilizce (�ok iyi)\\
Almanca (Az)\\
\end{tabular}

\vspace{0.05cm}
\noindent
{\bf ALINAN �D�LLER}

\noindent
\begin{tabular}{@{}ll@{}}
S�leyman Demirel �niversitesi & Elektronik ve Haberle�me M�hendisli�i \\
				    		  & B�l�m Birincili�i\\
TOBB Ekonomi ve Teknoloji �niversitesi & Y�ksek �eref ��rencisi \\
									   & Fen Bilimleri Enstit�s� Birincili�i (2008) \\ 
									   & (4.0/4.0) \\
\end{tabular}

\vspace{0.05cm}
\noindent
{\bf SERT�F�KALAR}

\noindent
\begin{tabular}{@{}lll@{}}
{\bf Y{\i}l} & {\bf Sertifika} & {\bf Yer}\\
2013 & OpenStack Summit 2013 & Portland, Oregon, USA\\
2009 & Enterasys Wireless AP \& Controller Administrator & Ankara \\
2009 & Enterasys B \& C Series Switching and Routing & Ankara\\
2009 & Enterasys X Series Backbone Switching & Ankara \\
2008 & NetApp Storage Administrator & �stanbul \\
2006 & Solaris Security Administrator & Ankara \\
2006 & Solaris Network Administrator & Ankara \\
2006 & Solaris System Administrator & Ankara \\
2005 & Allied Telesyn Int. Certified System Engineer & Ankara \\
2004 & Cisco Router & Ankara\\
2004 & Cisco Network Essentials & Ankara \\
\end{tabular}


\vspace{0.05cm}
\noindent
{\bf �ALI�MA ALANLARI}

\noindent
\begin{tabular}{@{}ll@{}}
{\bf Sistem Y�netimi} & Linux (Debian, Ubuntu, Red Hat, CentOS, Fedora) \\
				& Solaris, FreeBSD \\
				& Windows Server 2003, 2008 ve 2012 \\
{\bf Temel Sistem Servisleri} & E-posta (Postfix, Zimbra) \\
				& Web Server (Apache, Nginx, PHP, MemCache) \\
				& DNS, DHCP, FTP (ISC Bind, ISC DHCP, ProFTP) \\
				& Uygulama Sunucular� (Tomcat, JBoss, Glassfish) \\
				& Radius, LDAP (Free Radius, OpenLDAP) \\
				& Proxy, ��erik Filtreleme (Squid, Dansguardian) \\
				& Yaz�l�m Geli�tirme (Maven, SVN, GIT, ANT, JIRA)\\
\end{tabular}
	
\begin{tabular}{@{}ll@{}}
{\bf �li�kisel Veritabanlar�} & Oracle, MySQL, PostreSQL, MsSQL \\
{\bf Bulut Bili�im}   & OpenStack, CloudStack, OpenNebula \\
				& IaaS, PaaS ve SaaS teknolojileri \\
{\bf Da��t�k Hesaplama Sistemleri} &	Y�ksek ba�ar�ml� ve k�me hesaplama \\
				& Hadoop, Cassandra \\
				& CPU ve GPU tabanl� sistemler \\
				& Paralel programlama (MPI, OpenMP) \\
{\bf Da��t�k Dosya Sistemleri} & Lustre, Ceph, GlusterFS, Microsoft DFS \\
{\bf Orkestrasyon Ara�lar�} & Puppet, Chef, Juju, Cfengine, Vagrant, Fuel \\
{\bf Bare Metal Provizyonlama} & Cobbler, Kickstart, MaaS, Viper \\
{\bf Sanalla�t�rma} & KVM, libvirt, Xen, VMWare, Hyper-V, Proxmox \\
{\bf Transmisyon Altyap�s�} & Infiniband, 40G/10G/Gigabit Ethernet \\
{\bf �nternet Eri�imi} & ATM, Metro Ethernet, Frame Relay, ADSL \\
{\bf LAN \& WAN} & Omurga ve kenar anahtar \\
				& Router ve routing teknolojileri \\
				& L2-L7 Anahtar  \\
				& G�venlik ara�lar� (Firewall, IDS/IPS, VPN) \\
{\bf A� Sanalla�t�rma} & OpenVswitch, Quantum \\
{\bf A� �zleme Ara�lar�} & MRTG, Cacti, Nagios, Ganglia \\
{\bf Uzaktan E�itim} & Open Meeting, Moodle, Adobe Connect \\
{\bf Depolama Teknolojileri} & SAN, NAS, Object Storage\\
				& Swift, NFS, iSCSI, SRP, IPoIB \\
{\bf Nesne Tabanl� Programlama} & Java, PHP, Python, Delphi, C, Javascript \\
{\bf Donan�m Bilgisi} & Sun, IBM, Fujitsu, HP, Dell, Huawei Sunucular \\
				& Cisco, Juniper, Enterasys, Huawei a� cihazlar� \\
				& Mellenox infiniband core cihazlar� \\
				& Veri depolama (NetApp, Oracle, Dell, HP) \\
				& Juniper Sald�r� Tespit ve �nleme cihazlar� \\
\end{tabular}


\vspace{0.05cm}
\noindent
{\bf HOB�LER}

\noindent
\begin{tabular}{@{}l@{}}
Otomobiller, ak�ll� telefonlar, android i�letim sistemi, elektrikli ara� tasar�m�
\end{tabular}


\vspace{1.05cm}
\noindent
{\bf YAYINLAR}

\noindent
H. Cotuk, B. Tavli, K. Bicakci, M.B. Akgun, "The Impact of Bandwidth Constraints on the Energy Consumption of Wireless Sensor Networks", Wireless Communications and Networking Conference (WCNC), 2014 IEEE, pp.2787,2792, 6-9 April 2014  \\ \\
H. Cotuk, O. Bektas, B. Caliskan, "Operating Cloud: NREN's Case", Terena Networking Conference, Dublin, May 2014  \\ \\
H. Cotuk, K. Bicakci, B. Tavli, E. Uzun, "The Impact of Transmission Power Control Strategies on Lifetime of Wireless Sensor Networks", IEEE Transactions on Computers, vol.63, no.11, pp.2866,2879, Nov. 2014. \\ \\
H. Cotuk, A. Omercioglu, N. Erginoz, "IEEE 802.1x, Radius and Dynamic VLAN Assignment", inet-tr'06, T�rkiye'de Internet Konferans�, Ankara, 2006. \\ \\




\end{document}
